\usepackage[utf8]{inputenc}
\usepackage[T1]{fontenc}
\usepackage[top=2cm, bottom=2cm]{geometry}
\usepackage{float}%positionner les images
\usepackage{graphicx}%pour les images
\usepackage{soul}
\usepackage{enumitem}
\usepackage{makecell}
\usepackage{url}
\usepackage{tablefootnote}
\usepackage[colorlinks = true,
linkcolor = blue,
urlcolor  = blue,
citecolor = blue,
anchorcolor = blue
hyperfootnotes=false]{hyperref}

 \newcommand{\HRule}{\rule{\linewidth}{0.5mm}} %------- Faire les trai Dans La Page De Garde

%---- Begin Tableau Considérations pour les systèmes relationnels vs NoSQL
\usepackage{array,multirow,makecell} 
\newcolumntype{C}[1]{>{\centering\arraybackslash }b{#1}}
\newcolumntype{F}[1]{>{ \centering \vspace{1.mm} \arraybackslash}m{#1}<{ \vspace{1.mm}\arraybackslash }}
\newcolumntype{R}[1]{>{  \vspace{1.mm} \arraybackslash}m{#1}<{ \vspace{1.mm}\arraybackslash }}
\newcolumntype{G}[1]{>{ \centering \vspace{2.mm} \arraybackslash}m{#1}<{ \vspace{2.mm}\arraybackslash }}
%---- End Tableau Considérations pour les systèmes relationnels vs NoSQL

%------- Begin Ajouter les subsubsection à la table des matiere--------
\addtocounter{tocdepth}{3}
\setcounter{secnumdepth}{3}
%------- end Ajouter les subsubsection à la table des matiere--------

\usepackage{footnote} %------ Pur ressortir les note des élèment du tableau

\usepackage[french]{babel}

\usepackage{fancyhdr}

\usepackage{adjustbox}% pour orionté le tab 90°
\usepackage{amssymb}% pour chekmark dans le tableau