\documentclass[11pt,a4paper,onecolumn,openright,oneside]{report}
\usepackage[square,sort&compress,comma,numbers]{natbib}
\usepackage[sectionbib]{chapterbib}
\usepackage[printonlyused,withpage]{acronym}
\usepackage[utf8]{inputenc}
\usepackage[T1]{fontenc}
\usepackage[top=2cm, bottom=2cm]{geometry}
\usepackage{float}%positionner les images
\usepackage{graphicx}%pour les images
\usepackage{soul}
\usepackage{enumitem}
\usepackage{makecell}
\usepackage{url}
\usepackage{tablefootnote}
\usepackage[colorlinks = true,
linkcolor = blue,
urlcolor  = blue,
citecolor = blue,
anchorcolor = blue
hyperfootnotes=false]{hyperref}

 \newcommand{\HRule}{\rule{\linewidth}{0.5mm}} %------- Faire les trai Dans La Page De Garde

%---- Begin Tableau Considérations pour les systèmes relationnels vs NoSQL
\usepackage{array,multirow,makecell} 
\newcolumntype{C}[1]{>{\centering\arraybackslash }b{#1}}
\newcolumntype{F}[1]{>{ \centering \vspace{1.mm} \arraybackslash}m{#1}<{ \vspace{1.mm}\arraybackslash }}
\newcolumntype{R}[1]{>{  \vspace{1.mm} \arraybackslash}m{#1}<{ \vspace{1.mm}\arraybackslash }}
\newcolumntype{G}[1]{>{ \centering \vspace{2.mm} \arraybackslash}m{#1}<{ \vspace{2.mm}\arraybackslash }}
%---- End Tableau Considérations pour les systèmes relationnels vs NoSQL

%------- Begin Ajouter les subsubsection à la table des matiere--------
\addtocounter{tocdepth}{3}
\setcounter{secnumdepth}{3}
%------- end Ajouter les subsubsection à la table des matiere--------

\usepackage{footnote} %------ Pur ressortir les note des élèment du tableau

\usepackage[french]{babel}

\usepackage{fancyhdr}

\usepackage{adjustbox}% pour orionté le tab 90°
\usepackage{amssymb}% pour chekmark dans le tableau
\makesavenoteenv{tabular}

\begin{document}
	\setcellgapes{3pt}
	
	\begin{titlepage}
		
		\begin{center}
			
			\textsc{\LARGE République Algérienne Démocratique et Populaire}\\
			\textsc{\LARGE Ministère de l’Enseignement Supérieur et de la Recherche Scientifique}\\
			\textsc{\LARGE Université Mouloud Mammeri de Tizi-Ouzou}\\
			
			\begin{figure}[H]
				\centering
				\includegraphics[scale=1.5]{ummto.PNG}
			\end{figure}
			
			\textsc{\Large En vue de l’obtention du diplôme de Master Academique}\\
			\textsc{\Large Spécialité : Informatique}\\
			\textsc{\Large Options : Systèmes Informatiques et Ingénierie Des Systèmes d’information}\\[1cm]
			
			
			\textsc{\Large Thème}\\
			\HRule \\[0.5cm]
			% { \huge \bfseries Mise en place d'une solution web ERP \\
			% 	Cas: Œuvres Universitaires \\[0.4cm] }
            { \huge \bfseries Implémentation d'une solution web ERP \\
				Cas: Œuvres Universitaires \\[0.4cm] }
			
			\HRule \\[0.5cm]
			
			
			\emph{Présenté par :}\\
			
			Yanis \textsc{OUERDANE}\\
			Mohand \textsc{OURAD}\\[1.5cm]
			
			\emph{Devant le jury composé de :}\\
			
			\leftskip=4.5cm
			Président(e) : M$^{me}$ / M$^{elle}$ / M$^{r}$ JOHN Doe \\
			Examinateur(trice) : M$^{me}$ / M$^{elle}$ / M$^{r}$ DOE John\\
			Promoteur(trice) : M$^{me}$ \textsc{Goumeziane} Lynda \\ [1.5cm]
			
			
			\vfill
            {\large Année universitaire: 2020/2021}
			
		\end{center}
		
	\end{titlepage}

	\chapter*{\huge Remerciements}
	
	\begin{center}
		\it \Large
		D’abord, nous remercions le bon \textbf{DIEU} de nous avoir donné santé et courage pour réaliser ce travail.\\
		
		Nous tenons à exprimer notre profonde gratitude à notre encadreur \textbf{Mme GOUMEZIANE Lynda}, pour nous avoir encadré et guidé et surtout pour ses judicieux conseils qui ont contribué à alimenter notre réflexion.\\
		
		Nous remercions chaleureusement les membres de jury pour l’honneur qu’ils nous ont fait en acceptant de juger notre travail.\\
		
		Nos sincères sentiments vont à nos parents qui ont sacrifié jusqu’aujourd’hui et leurs encouragements tout le long de notre parcours.\\
		
		\leftskip=10cm 
		
		Yanis, Mohand.
		
		\leftskip=0cm
		
	\end{center}
	%------------------- End Remerciement ----------------------------
	
	\chapter*{\huge Dédicaces}
	
	\begin{center}
		\it \Large
		Je dédie ce modeste travail :
		A mes très chers parents que dieu les
		protègent, pour leur aide et leur soutien tout au long
		de mes études,\\
		
		A toute ma famille, à mes chers amis,\\
		
		Enfin à tous ceux qui ont contribué de près
		ou de loin pour la réalisation de ce travail.\\
		
		\leftskip=12cm
		
		Yanis.
		
		\leftskip=0cm
		
	\end{center}
    \chapter*{\huge Dédicaces}
	
	\begin{center}
		\it \Large
		Je dédie ce modeste travail :
		A mes très chers parents que dieu les
		protègent, pour leur aide et leur soutien tout au long
		de mes études,\\
		
		A toute ma famille, à mes chers amis,\\
		
		Enfin à tous ceux qui ont contribué de près
		ou de loin pour la réalisation de ce travail.\\
		
		\leftskip=12cm
		
		Mohand.
		
		\leftskip=0cm
		
	\end{center}
	
	\tableofcontents
	\listoffigures
	\addcontentsline{toc}{chapter}{Table des figures}
	\listoftables
	\addcontentsline{toc}{chapter}{Liste des tableaux}

	\section*{\Huge{Liste des Abréviations}}
\addcontentsline{toc}{chapter}{Liste des Abréviations}

\vspace{2em}

    \begin{acronym}
        \leftskip=1.5em

        \acro{ERP}{Enterprise Ressource Planning}
        \acro{PGI}{Progiciel de Gestion Intégré}
        \acro{MRP}{Manufacturing Resource Planning}
        \acro{PME}{Petite ou moyenne entreprise}
        \acro{ETI}{Entreprise de taille intermédiaire}
        \acro{SGBD}{Système de gestion de base de données}
        
        \acro{D.O.U}{Diréction des Œuvres Universitaires}
        \acro{TIC}{Les Technologies de l’Information et de la Communication}

        \acro{UML}{Unified Modeling Language/Langage de Modélisation Unifié}
        
        \acro{PERN}{PostgreSQL, ExpressJS, React et NodeJS}
        \acro{MERN}{MongoDB, ExpressJS, React et NodeJS}
        \acro{CRUD}{Create, Read, Update, Delete}
        \acro{SQL}{Structured Query Language}
        \acro{NoSQL}{Not only SQL}
        \acro{MVC}{Model View Controller}
        \acro{MEVN}{MongoDB, ExpressJS, Vue et NodeJS}
        \acro{MEAN}{MongoDB, ExpressJS, Angular et NodeJS}
        \acro{JWT}{JSON Web Token}
        
    \end{acronym}

	\pagestyle{fancy}
	\fancyhead{}
	
	\renewcommand{\chaptermark}[1]{\markboth{\bsc{\chaptername~\thechapter{} :} #1}{}}
	
	% \rhead[\textsl{\leftmark}]{\textsl{\rightmark}}
	\lhead[\textsl{\rightmark}]{\textsl{\leftmark}}
	
	\renewcommand{\headrulewidth}{1.2pt}
	
	\newcommand\blfootnote[1]{
		\begingroup
		\renewcommand\thefootnote{}\footnote{#1}
		\addtocounter{footnote}{-1}
		\endgroup
	}

	\part{Introduction Générale}
		Actuellement, le monde connaît une avancée technologique considérable dans tous les secteurs et cela grâce à l'informatique qui est une science qui étudie les techniques du traitement automatique de l'information. Elle joue un rôle important dans le développement de l'entreprise et d'autres établissements (ex : administrations hospitalières).\\

		Avant l'invention de l'ordinateur, on enregistrait toutes les informations manuellement sur des supports en papier ce qui engendrait beaucoup de problèmes tel que la perte de temps considérable dans la recherche de ces informations ou la dégradation de ces dernières ...Etc.\\
		
		Ainsi, jusqu'à présent, l'ordinateur reste le moyen le plus sûr pour le traitement et la sauvegarde de l'information. Cette invention a permis d'informatiser les systèmes de données des entreprises, ce qui est la partie essentielle dans leur développement aujourd'hui surtout depuis l'apparition des nouvelles technologies, notamment les \acs{ERP}.\\
		
		L'\acs{ERP} (Entreprise Ressource Planning) ou Progiciel de Gestion Intégré en français est une solution informatique destinée au pilotage des structures et entreprises. Son fonctionnement est basé sur le regroupement d'un ensemble de tâches liées aux activités d'une société.\\
		
		L'\acs{ERP} est capable de couvrir une large palette de gestion d'une entreprise allant de la gestion comptable à la gestion commerciale sans oublier la gestion de la paye ou encore la gestion des stocks. L'histoire des logiciels \acs{ERP} a permis au fin des époques de doter cet outil d'une couverture fonctionnelle toujours plus stratégique.\\
		
		C'est dans ce cadre que s'inscrit notre projet fin d'études qui a pour objectif de mettre en place un module de gestion des œuvres universitaires (que ce soit de la restauration, des bourses, d'hébergement ou encore des transports). Pour cela nous avons organisé notre travail en quartes chapitres :\\
		
		\begin{itemize}
			\item Le deuxième chapitre intitulé \textbf{"Les œuvres universitaires"}, est consacré à la présentation des œuvres universitaires et l'impact des \acs{ERP} sur celles-ci.
			\item Le premier chapitre intitulé \textbf{"Les Solutions de Gestion d'Entreprise"}, est dédié à la description du concept de l'entreprise et les problématiques des entreprises et les solutions qu'offre un \acs{ERP} à celles-ci.
			\item Le troisième chapitre intitulé \textbf{"Analyse et Conception"}, présente les étapes de la conception et la modélisation de notre projet.
			\item Le quatrième chapitre intitulé \textbf{"Réalisation"}, dans ce chapitre nous présenterons l'environnement et les outils utilisés pour le développement, à la fin nous allons présenter quelques interfaces et le résultat obtenu.
		\end{itemize}
	
	\part{Définitions Générales}
		\chapter{Les Progiciels de Gestion Internes}

\section{Introduction}
Le but de ce chapitre est de présenter globalement le progiciel de gestion interne aussi appeler \acs{ERP}.\\

Dans un premier temps, nous définirons le concept d'\acs{ERP} et son évolution dans le temps.\\

Par la suite, nous aborderons les avantages liés a l'intégration d'un tel système dans les deux aspects administratif et opérationnel ainsi que ses inconvénients.\\

Nous finirons avec les multiples fonctionnalités de l'\acs{ERP}.\\

\section{Définition}
Le sigle \acs{ERP} veut dire Entreprise Resource Planning son semblable en Français est Progiciel de Gestion Intégré abrévié PGI.\\

Contrairement au MRP qui se contente de la planification des besoins, l’\acs{ERP} est un logiciel qui permet la gestion de l’ensemble des sous-systèmes d’une entreprise ainsi que la coordination de ceux-ci.\\

Pour y parvenir l’\acs{ERP} intègre l’ensemble des fonctions utiles d’une entreprise sous forme de modules qui partagent une base de données unique, ceci permet l’échange d’informations entre les modules, dans ce cas-là on parle de moteur de Workflow.

\section{Historique}
La création de l’\acs{ERP} revient principalement à Joseph Orlicky qui créa dans les années 1960 le MRP abréviation de Material Requirements Planning ancêtre de l’\acs{ERP}, le MRP répond essentiellement aux besoins de planification des entreprises.\\

La notion d’\acs{ERP} tel que nous la connaissons à fait son apparition dans les années 90, mais n’a connu son essor que dans les années 2000 avec l’arrivée de l’internet, l’utilisation de l’\acs{ERP} se généralise et évolue jusqu’à arriver à l’\acs{ERP} tel que nous le connaissons aujourd’hui.

\section{Avantages liés à l’intégration d’un \acs{ERP}}
Les bénéfices liés à l’implémentation d’un ERP ont été prouvé par bon nombre de recherches, l’une d’elles mener par le groupe Aberdeen qui ont quantifié et publié les résultats suivants :\\

\begin{itemize}
    \item Réduction des coûts d’opérations de 22\%
    \item Réduction des coûts d’administration de 20\%
    \item Réduction d’inventaires de 17\%
    \item Amélioration du temps de livraison de 19\%
    \item Amélioration du respect des délais et des budgets de 17\%\\
\end{itemize}

Même les entreprises en difficulté ont réalisé des bénéficies grâce à l’intégration de leur ERP, leurs résultats s’élèvent à :\\

\begin{itemize}
    \item Réduction des coûts d’opérations de 7\%
    \item Réduction des coûts d’administration de 4\%
    \item Réduction d’inventaires de 9\%
    \item Amélioration du temps de livraison de 11\%
    \item Amélioration du respect des délais et des budgets de 6\%\\
\end{itemize}

Comme l’étude le souligne le gain en pourcentage ne paraît pas impressionnant, mais pour chaque million de dollars déboursé dans les coûts d’opérations, 70 000 \$ sont économisé.\\

En effet nous pouvons constater les gains en productivité et en maturité des entreprises, pour y parvenir l’\acs{ERP} procède à une amélioration sur plusieurs aspects :

\subsection{Aspect administratif}
En fusionnant tous les systèmes de l’entreprise en une seule application, l’installation de l’\acs{ERP} conduit à une réduction des coûts d’exploitation et de maintenance, et comme l’\acs{ERP} possède une architecture sous forme modulaire il fournit une infrastructure qui assure une flexibilité en cas de changements futurs, donc offre la possibilité d’implémenter de nouvelles fonctionnalités.\\

Une seule application donc une seule base de données, cette base de données unique permet un gain de temps. Réduire le volume d’information inutile et d’éviter les saisies multiples, donc l’installation de l’\acs{ERP} permet la résolution des problèmes d’incohérence des informations et rend les données enregistrées plus fiables.\\

De plus cela évite les activités manuelles de traitement, de comparaison et de recherche réaliser par les employés dans le cadre de l’interfaçage des différents services. Ce qui conduit à un gain de temps et une croissance de la productivité administrative.

\subsection{Aspect opérationnel}
L’utilisation d’un \acs{ERP} conduit à la suppression des risques opérationnels et aux risques de pertes liés à des erreurs humaines où des dysfonctionnements dans le contrôle interne, fraudes qui peuvent résulter d’un dysfonctionnement des systèmes d’information déjà en place, l’\acs{ERP} permet la pertinence des informations partagées et évite ces dysfonctionnements qui peuvent être plus ou moins grave et qui peuvent entrainer des coûts supplémentaires inutile.\\

L’\acs{ERP} permet aussi un suivie au niveau des achats jusqu’aux ventes. En effet, dès la création de la commande, des données telles que le calcul des marges et des crédits est généré automatiquement de façon dynamique réalisant ainsi une intégration financière, grâce à cette fonctionnalité, l’\acs{ERP} aide les dirigeants dans la planification et la prise de décision, et leur permet d’améliorer la gestion des ressources, et ainsi améliorer les décisions opérationnelles.\\

De plus la centralisation est très bénéfique pour les services de finance. Car ce logiciel permet également la centralisation des tâches qui permet à une amélioration de la productivité en réduisant le nombre du personnel qui travaille sur la même tâche, cela permet des économies d’échelles notamment en matière de facturation.

\section{Inconvénients}
L’ERP offre des avantages non négligeables, mais une telle solution doit forcément comporter quelques désavantages.\\

Les projets ERP conduisent généralement à des coûts lors de la mise en place et de la maintenance. De plus la complexité des programmes utilisés requiert l’utilisation et l’entretien de serveurs puissants. Ce qui implique que le coût est souvent dépassé comme le montre l’étude CXP 2017.\\

\begin{figure}[H]
    \centering
    \includegraphics[scale=0.4]{ERP/graph-depassement-budget.jpg}
    \caption{Taux du dépassement de budget lors de l’implémentation d’un ERP}
\end{figure} 

On peut constater qu’en 2017 plus de 60\% des entreprises qui ont implémenté un ERP ont dépassé le budget prévu, 58\% en 2016 et 55\% en 2015.\\

En outre du coût, un projet de cette envergure requiert un temps et des ressources qui peuvent dépasser les prévisions, tel que le montre l’étude ERP Report 2010 du cabinet de conseil Panorama Consulting.\\

\begin{figure}[H]
    \centering
    \includegraphics[scale=0.65]{ERP/graph-taux-implementation.png}
    \caption{Taux des dépassements des délais lors de l’implémentation d’un ERP}
\end{figure} 

Cette étude montre que plus de 35.5\% des entreprises ayant mis en place un ERP, ont vu le temps prévu pour l’implémentation dépassée, il est aussi à noter que la durée moyenne de la mise en place d’un ERP est de 18,4 mois qui varient d’un éditeur à un autre.

\section{Fonctionnalités}
L’ERP gère et organise les informations de l’ensemble des services de l’entreprise de façon automatique et dynamique, de l’achat des ressources à la vente en passant par la production, ces fonctionnalités sont nombreuses. Les modules plus couramment utilisés sont les suivants : \\

\begin{itemize}
    \item Gestion d’achats
    \item Gestion de la chaine logistique
    \item Gestion de stock et d’inventaire
    \item Gestion de production
    \item Gestion de projet
    \item Gestion des ressources humaines 
    \item Gestion comptabilité
    \item Gestion commerciale
    \item CRM : Gestion des relations clients\\
\end{itemize}

Chaque module couvre des fonctionnalités qui lui sont propres, dans le tableau ci-dessous une présentation de certains modules et des fonctionnalités qu’ils proposent.

\begin{table}[H]
    \begin{center}
        
        \begin{tabular}{|F{4cm}|R{10cm}|}
            \hline
            \textbf{Modules}  & \makecell[c]{\textbf{Fonctionnalités}} \\
            \hline
            Achats
            &
            Gestion de toutes les transactions comptables, telle que les bons de commande pour l’approvisionnement. Etc.\\
            
            \hline
            Chaine logistique
            &
            Gestion des ressources utilisées pour le pilotage de la chained’approvisionnement et de livraison.\\
            
            \hline
            Stock
            &
            Gestion des mouvements du stock, état du stock, entreposage.\\
            
            \hline
            Production
            &
            La gestion de la production, permet de réguler l’offre et les besoins en
            ressources par apport à la demande, impliquent la planification des ordres
            de fabrication et le contrôle de qualité.\\
            
            \hline
            Gestion de projet
            &
            Gestion de l’ensemble des projets de l’entreprise, de ces tâches et de ces plannings.\\
            
            \hline
            Ressources humaines
            &
            Gestion des ressources humaines et l’organisation de la rémunération des employés ainsi que des plannings de travail de ceux-ci.\\
            
            
            \hline
            Comptabilité
            &
            Gestion des obligations comptable auxquelles l’entreprise est soumise et suivie en temps réel de la santé financière de celle-ci, ainsi que de la gestion de facturation et des multidevises.\\
            
            \hline
            Commerciale
            &
            Gestion de l’aspect commerciale de l’entreprise, permet la gestion de l’ensemble des commandes clients et de leur facturation, permet aussi la réalisation de devis rapide et précise.\\
            
            \hline
            CRM
            &
            Gestion des relations clients, permet de réaliser de meilleurs suivis de
            l’environnement : clients, fournisseurs, prospects. etc.\\
            
            
            \hline
        \end{tabular}	
        \caption{Les Modules d'un ERP et leurs fonctionnalités}
    \end{center}
\end{table}

\section{Conclusion}
Après avoir défini lors de la première partie le concept d’entreprise, l’environnement et les contraintes auxquelles elle fait face. L’obligation de dégager un bénéfice est vitale, mais un certain nombre de points complexifient leurs systèmes d’information et freine la croissance économique de celle-ci, ces mêmes points qui rendent le recours à une technologie de l’information telle que l’ERP presque obligatoire.\\

Dans le deuxième point nous avons étudié l’ERP qui est au cœur du système d’information et qui permet la gestion de tous les services de l’entreprise. Ainsi que les avantages et les inconvénients à recourir à un tel outil.\\

Ensuite nous allons approfondir la recherche et étudier lors du deuxième chapitre, la gestion de stocks et d’approvisionnement.


\newpage

\leftskip=0cm
\renewcommand{\bibname}{Référence bibliographique et webographique du chapitre 1}
\bibliographystyle{ieeetr}	
\bibliography{ERP/erp}
		\chapter{L'Entreprise et les Œuvres Universitaires}

\section{Introduction}
Dans ce chapitre nous commencerons par aborder le concept d'entreprise, une définition générale de celle-ci, son environnement économique et l'évolution des technologies de l'information et de la communication.\\

Nous passerons par la suite à la présentation de l'organisme d'accueil qu'est la direction des œuvres universitaires (D.O.U) en général, son affiliation à l'office National des œuvres universitaires, ses missions et ses activités et son organisation administrative. 


Notre projet comprend la conception et la réalisation d'une application d'aide à la gestion des ressources de la direction des œuvres universitaires (D.O.U) en général. Celle-ci offre des services de transport, de restauration, de bourse, d'hébergement et d'activités scientifiques, culturelles et sportives. Ces mêmes services lui ont été délégués par l'Office National des œuvres universitaires en fonction de la wilaya où est siégée cette même \acs{D.O.U}.\\

À cet effet, il est nécessaire de définir un concept général d'une entreprise et de  présenter les services, en plus des missions, de la \acs{D.O.U} en tant qu'organisme d'accueil afin de comprendre leur principale activité.\\

\section{Concept d’Entreprise}
\subsection{Définition de l’entreprise }
Une entreprise\cite{def-entreprise} est un groupe d'unités légales qui se combinent pour créer une unité organisationnelle dont le but est de produire des biens ou des services. Il jouit d'une autonomie de décision dans l'affectation et l'utilisation de ses ressources disponibles.\\

Selon l'aspect économique, une entreprise est une unité qui produit des biens matériaux de consommation, de la matière première ou des services. Selon l'aspect sociologique cette unité est une structure avec des dirigeants, des salariés et des investisseurs.\\ 

Comme tout ensemble, chaque entité de cette unité à des intérêts qui peuvent différer des autres membres de l'entreprise. D'un côté, les investisseurs se concentrent plus sur le rendement financier et des marges bénéficiaires sur le retour de leurs investissements, d'un autre côté, les dirigeants ont tendance à favoriser la performance, la croissance et la productivité de l'entreprise tout en s'assurant du bon contrôle et de la bonne gestion des salariés en pensant à minimisant les couts et les dépenses, tandis que les salariés, se focalisent sur leurs objectifs de réussite personnelle et professionnelle tout en s'assurant de bien accomplir leurs missions respectives.\\

Une entreprise publique, est une entreprise dont l'État dispose une part majoritaire du capital ou des voix attachées aux parts émises. L'État a donc le pouvoir d'exercer un contrôle direct ou indirect avec une influence dominante sur les décisions de cette dernière\cite{def-entreprise-pub}, par définition les \acs{D.O.U} prennent partie dans cette forme d'entreprise.\\

\subsection{Environnement Économique}
L'environnement économique d'une entreprise\cite{env-entreprise} est un concept très large, il rassemble toutes les facteurs externes à celle-ci qui rentrent en rapport explicitement ou implicitement avec elle de manière a influencer les décisions de l'entreprise elle même. Les facteurs en question:\\

\begin{itemize}
    \item Le facteur démographique.
    \item le facteur économique.
    \item le facteur sociologique.
    \item le facteur technologique.
    \item le facteur politique et légale.
    \item le facteur écologique.
    \item le facteur de la concurrence et des produits de substitution.\\
\end{itemize}

Tous ces facteurs, participent donc, de près ou de loin, à la performance de l'entreprise dans son environnement\cite{perf-entreprise}. Les entreprises concurrentes voulant optimiser leurs processus et minimiser leurs dépenses se sont tourné vers les nouvelles technologies dont les systèmes d'informations, de gestion et de la communication.\\

\section{La Direction des Œuvres Universitaires}
\subsection{Présentation de la \acs{D.O.U}}
Les directions des œuvres Universitaires\cite{dou} ont été créé conformément à l'arrêté interministériel du 22 décembre 2004 comportant la fixation de leurs sièges en plus de la liste constituant les résidences universitaires qui leur sont rattachées. Elles sont placées sous la tutelle de l'Office National des œuvres universitaires.\\

Elles sont chargées de veiller à la gestion des ressources financières et humaines, du bon déroulement et du contrôle des résidences universitaires dont elles sont responsables, de la gestion du transport entre les résidences et les différents établissements de l'enseignement supérieur et de la restauration, de la wilaya dont elles font partie.\\

\subsection{Missions et activités de la \acs{D.O.U}}
Sa mission est de prendre en charge les différentes activités qui lui sont déléguées par l'Office National des œuvres universitaires\cite{onou-arrete} qui est lui-même sous la tutelle du ministère de l'enseignement supérieur et de la recherche scientifique.\\

Principalement organiser et gérer les services d'hébergement, de restauration, de bourse, de transport et activités scientifique, culturelles et sportives, de manière à assurer la satisfaction des besoins de l’étudiants.\\

Plus précisément :

\begin{itemize}\renewcommand{\labelitemi}{$\bullet$}
    \item Veiller à la gestion des moyens matériels et financiers qui lui sont affectés.
    \item Prendre les mesures nécessaires au bon fonctionnement des structures placées sous son autorité.
    \item Veiller à la gestion de son personnel et du personnel des résidences universitaires sous son autorité.
    \item Veiller au bon contrôle rationnelle des moyens mis a la disposition des résidences universitaires sous son autorité.
    \item S'assurer, avec les structures et organismes concernés, du suivi des opérations d'investissement et d'équipement des résidences universitaires sous son autorité.
    \item Soumettre périodiquement des rapports sur le fonctionnement des résidences universitaires sous son autorité.
    \item Participer à la création et au bon suivi de l'application du règlement intérieur des résidences universitaires sous son autorité.
    \item Approuver et suivre le bon déroulement des programmes d'activités scientifiques, culturelles et de loisirs des résidences universitaires sous son autorité.
    \item Passez tout marché et contrat en relation avec la restauration et le transport assuré par les résidences universitaires sous son autorité.
    \item Exercer l'autorité hiérarchique sur son personnel.
    \item Nommer les personnels dont le mode de nomination n'est pas prévu.
    \item Ordonner les crédits qui lui sont délégués.
\end{itemize}

\subsection{L’organisation de la \acs{D.O.U}}
La direction des œuvres universitaires\cite{onou-arrete} est composée de 04 départements selon le diagramme suivant :

\begin{figure}[H]
    \centering
    \includegraphics[scale=0.6]{DOU/direction-org.jpg}
    \caption{Organigramme de La Direction des Œuvres Universitaires}
\end{figure}

Chaque département regroupe plusieurs services qui sont chargé d'assurer différentes fonctions :

\subsubsection{Le département du contrôle et de la coordination}
\begin{itemize}
    \item Service du transport.
    \item Service de la restauration.
    \item Service de l'hébergement.
    \item Service des activités scientifiques, culturelles et sportives.\\
\end{itemize}

Ces différents services ont plusieurs taches différentes dont ils sont chargés d'exécuter et de veiller sur leur bonne exécution:

\paragraph{Hébergement}
Le service d'hébergement comprend deux sections, la section de l'attribution de l'hébergement et la section de la gestion. À se fait, ce service a pour tâches :

\begin{itemize}
    \item Inscription des nouveaux bacheliers, et réinscription des anciens étudiants, ceci ce fait au niveau de chaque résidence.
    \item Contrôler les dossiers.
    \item Établir des statistiques sur l'état des résidences en rédigeant des listes globales de tous les étudiants et leurs répartitions par résidence et par bloc en tenant compte des nombres de places libres, les abandons, ... etc.
\end{itemize}

\paragraph{Transport}
\begin{itemize}
    \item Assurer le transport des étudiants des résidences universitaires vers les campus pédagogiques en tenant compte du trajet inverse.
\end{itemize}

\paragraph{Restauration}
\begin{itemize}
    \item Assurer les repas au étudiants internes et externes.
\end{itemize}

\paragraph{Bourse}
\begin{itemize}
    \item Assurer le traitement et le suivi des dossiers des étudiants bénéficiaires de bourses.
    \item Assurer le renouvellement des bourses.
    \item Assurer le paiement régulier des bourses.
    \item Assurer le traitement et la prise en charge des bourses des étudiants étrangers.\\
\end{itemize}

Ce département est chargé de :
\begin{itemize}\renewcommand{\labelitemi}{$\bullet$}
    \item Mettre en œuvre les plans de transport universitaire des résidences universitaires rattachées à la \acs{D.O.U} et superviser le processus jusqu'à son aboutissement.
    \item Superviser, surveiller et orchestrer les actes d'œuvre universitaires assurées par les résidences universitaires associées à la \acs{D.O.U}.
    \item Présenter des méthodes rationnelles d'utilisation de tous les moyens dédiés aux activités des œuvres universitaires.
    \item Contrôler, et Assurer la bonne application des programmes d'activités scientifiques, et sportives approuvées par le directeur de la direction.\\
\end{itemize}

\subsubsection{Le département des ressources humaines}
\begin{itemize}
    \item Service de la gestion des carrières.
    \item Service de la formation et de perfectionnement.\\
\end{itemize}

Ce département est chargé de :
\begin{itemize}\renewcommand{\labelitemi}{$\bullet$}
    \item La gestion de la carrière du personnel de la \acs{D.O.U}.
    \item L'implémentation des plans de formation et perfectionnement du personnel de la \acs{D.O.U}.
\end{itemize}

\subsubsection{Le département des bourses}
\begin{itemize}
    \item Service de l'attribution des bourses.
    \item Service du renouvellement des bourses.\\
\end{itemize}

Ce département est chargé de :
\begin{itemize}\renewcommand{\labelitemi}{$\bullet$}
    \item Suivre et garantir le traitement des dossiers des étudiants bénéficiaires de bourses.
    \item Garantir le paiement régulier des bourses.
    \item Prendre en charge et garantir le traitement des bourses des étudiants étrangers.
\end{itemize}

\subsubsection{Le département des finances et des marchés publics}
\begin{itemize}
    \item Service du budget et de la comptabilité.
    \item Service des marchés publics.
    \item Service du suivi des opérations de construction et de l'équipement.\\
\end{itemize}

Ce département est chargé de :
\begin{itemize}\renewcommand{\labelitemi}{$\bullet$}
    \item Gérer les moyens matériels et financiers qui ont été mis à la disponibilité de la \acs{D.O.U}.
    \item Garantir de service de traitements des personnels de la \acs{D.O.U}.
    \item Garantir le contrôle des divers étapes de passation des marchés publics et d'en surveiller l'exécution par les résidences universitaires.
    \item Garantir, en conjonction avec les services concernés, la surveillance des actes de construction et d'équipement des résidences universitaires.\\
\end{itemize}

\section{La Problématique}
Parmi les problèmes qui existent après l’analyse de ce système, on peut citer :
\begin{itemize}
    \item Le transfert de données entre la \acs{D.O.U} et les différentes résidences se fait manuellement.
    \item La répartition des étudiants d’une manière arbitraire, ce qui engendre l’augmentation de l’enveloppe budgétaire alloue au transport.
    \item La perte de temps.
    \item La non disponibilité des informations au bon moment.
    \item La \acs{D.O.U} dispose d'un réseau informatique à haut débit mais très mal exploité. En fait, il n'existe aucune application ou logiciel fonctionnant sous réseau.
    \item La grande difficulté d'accéder aux informations en tant qu'étudiants. Les seules moyens qui existent actuellement sont des pages facebook, des photos de fiches mal prise et des sites internet existant mais qui ne marches pas et/ou ne contient pas les informations pertinente dont l'étudiant a besoin.
    \item La non existence des informations concernant les repas des services de restauration.
    \item L'inconsistance et l'incohérence des informations.\\
\end{itemize}

\section{Présentation de la solution}
Afin de remédier aux problèmes cités ci-dessus. Nous avons proposé de concevoir une application de gestion de ressources en se basant sur les Progiciels de Gestion Intégré \acs{ERP}, ceci pour permettre de :\\

\begin{itemize}
    \item Informatiser l'ajout, la validation et le rejet des dossiers.
    \item Informatiser la liste des résidences et des campus universitaires.
    \item Automatiser la gestion des trajets du service des transports.
    \item Rendre public les informations sur les services de restauration et de transport notamment les menus des restaurants universitaires et les trajets des bus universitaires. 
    \item Informatiser l'affectation des étudiants aux résidences.
    \item Informatiser la gestion de la restauration, notamment les ingrédients, les plats, les menus et le calendrier des menus.
    \item Avoir une plateforme centraliser pour la gestion de tous les services du département du contrôle et de la coordination, voire plus.
    \item Avoir un rapport mensuel des menus, des ingrédients et des plats du service de la restauration.
    \item Avoir un rapport mensuel des trajets et des bus du service des transports.
    \item Normaliser et regrouper les informations concernant les \acs{D.O.U}.\\
\end{itemize}

\section{Conclusion}
Dans ce chapitre nous avons abordé les concepts de base d'entreprise, et l'organisme d'accueil qu'est la direction des œuvres universitaires.\\

Nous avons par la suite présenté les missions et les activités de l'organisme d'accueil tout en détaillant l'organisation départementale de ce dernier, en prenant compte des services et des tâches qui leur sont rattachées.\\

En dernier, nous avons retenu plusieurs problèmes qui sont liés particulièrement à une mauvaise gestion des ressources en plus d’une retenue de l’information et le traitement manuel de l'information, tout en citant une présentation simple d'une solution a ces derniers.\\

Le chapitre suivant sera consacrée à l'analyse plus détailler du système et des ses besoins, et une conception des solutions de notre application.

	\part{Analyse, Conception \& Réalisation}
		\chapter{Analyse \& Conception}

\section{Introduction}
Le but de ce chapitre est d'aborder les concepts essentiels a la réalisation du projet.\\

En première partie, nous ferons un rappel sur les généralités d'UML, sa définition et ses diagrammes.\\

Par la suite, nous passerons à l'analyse des principaux utilisateurs du système et les besoins de ces derniers.\\

Nous terminerons par une conception des diagrammes de classes et des diagrammes de séquences ainsi qu'une présentation de la structure de la base de données.\\

\section{Présentation d'\acs{UML}}
\acs{UML} est un acronym qui signifie Unified Modeling Langage ou Langage de Modélisation Unifié en français, est apparu pour la première fois dans les années 1990. En simple, \acs{UML} est une approche moderne et méthodique de la modélisation et de la documentation des applications. C'est l'une des techniques de modélisation des processus d'entreprise les plus populaires.\\

Son approche est basée sur des représentations schématiques des composants d'un logiciel. Cette représentations visuelles nous donnent la possibilité de mieux mesurer les éventuelles erreurs ou défauts des logiciels.\\


\subsection{Les diagrammes \acs{UML}}
\subsubsection{Diagrammes des cas d'utilisation}
\subsubsection{Diagrammes de classes}
\subsubsection{Diagrammes de séquence}

\section{Analyse}
La phase d’analyse débute par la spécification des besoins fonctionnels du système, ensuite définir les acteurs ainsi que leurs différentes tâches. Ceci nous mène vers un résultat qui est un diagramme de cas d’utilisation traduisant la dynamique système qui sera utilisé dans la phase de conception.

\subsection{Spécification des besoins}
\subsubsection{Besoins fonctionnels}
C’est une description des caractéristiques du système et de ses fonctionnalités. En partant de ce principe notre application doit permettre :

\begin{longtable}{|Z{3cm}|L{10cm}|}
    \caption{Les acteurs et leurs fonctions respectives}\\
    \hline
    \textbf{Acteur} & \textbf{Fonctions} \\
    \hline
    \endfirsthead
    \multicolumn{2}{c}%
    {\tablename\ \thetable\ -- \textit{Suite de la page précedente}} \\
    \hline
    \textbf{Acteur} & \textbf{Fonctions} \\
    \hline
    \endhead
    \hline \multicolumn{2}{r}{\textit{Suite dans la page suivante}} \\
    \endfoot
    \hline
    \endlastfoot
    Visiteurs &
        \textbf{Consultation de la page d’accueil}
        \begin{itemize}
            \item Consulter le service d'hébergements
            \item Télécharger le document d'aide à la constitution du dossier d'hébergement
            \item Consulter le service de restauration
            \item Consulter le service des transports
            \item Consulter le service des bourses
            \item Télécharger le document d'aide à la constitution du dossier de bourse
            \item Télécharger le document d'aide au renouvellement du dossier de bourse
            \item Télécharger le document d'aide au transfert du dossier de bourse
        \end{itemize} \\
        &
        \textbf{Consultation du calendrier des menus de chaque restaurant d'un mois défini}
        \begin{itemize}
            \item Changer le mois du calendrier des menus
            \item Changer l'affichage des menus par mois, semaine, jour de travail, jour et par agenda
            \item Voir plus d'informations sur un seul menu
        \end{itemize} \\
        &
        \textbf{Consultation du calendrier de tous les trajets de chaque Campus/Résidence d'un mois défini}
        \begin{itemize}
            \item Changer le mois du calendrier des trajets
            \item Changer l’affichage des trajets par mois, semaine, jour de travail, jour et par agenda
            \item Voir plus de détails sur un seul trajet
        \end{itemize}
     \\
     \hline
    & 
    En plus des fonctionnalités du visiteur il peut aussi:\\
    Administrateur &
    \textbf{Hébergement}
    \begin{itemize}
        \item Consulter la liste des dossiers d'hébergement
        \item Sélectionner un ou plusieurs dossiers d'hébergement
        \item Ajouter un dossier d'hébergement
        \item Supprimer un ou plusieurs dossier(s) d'hébergement
    \end{itemize}\\
    &
    \begin{itemize}
        \item Voir plus d'informations sur un dossier d'hébergement
        \item Valider un dossier d'hébergement
        \item Refuser un dossier d'hébergement
    \end{itemize}\\
    &
    \textbf{Bourse}
    \begin{itemize}
        \item Consulter la liste des dossiers de bourse
        \item Sélectionner un ou plusieurs dossiers de bourse
        \item Ajouter un dossier de bourse
        \item Supprimer un ou plusieurs dossier(s) de bourse
        \item Voir plus d'informations sur un dossier de bourse
        \item Valider un dossier de bourse
        \item Refuser un dossier de bourse
        \item Changer l'ordre d'affichage par attributs
    \end{itemize}\\
    Administrateur &
    \textbf{Utilisateurs}
    \begin{itemize}
        \item Consulter la liste des utilisateurs
        \item Sélectionner un ou plusieurs utilisateur(s)
        \item Ajouter un utilisateur
        \item Supprimer un ou plusieurs utilisateur(s)
        \item Voir plus d'informations sur un utilisateur
        \item Modifier les informations d'un utilisateur
        \item Filtrer par attributs
    \end{itemize}\\
    &
    \textbf{Campus \& Résidences}
    \begin{itemize}
        \item Consulter la liste des Campus \& Résidences
        \item Ajouter un campus ou une résidence
        \item Supprimer un campus ou une résidence
        \item Voir plus d'informations sur un campus ou une résidence
        \item Modifier les informations d'un campus ou une résidence
        \item Rechercher un campus ou une résidence
        \item Filtrer par campus seulement résidences seulement ou par campus \& résidences
    \end{itemize}\\
    &
    \textbf{Transport - Calendrier}
    \begin{itemize}
        \item Voir les trajets détaillé de tout le mois
        \item Ajouter un trajet
        \item Modifier un trajet
        \item Supprimer un trajet
    \end{itemize}\\
    &
    \textbf{Transport - Bus}
    \begin{itemize}
        \item Consulter la liste des bus
        \item Sélectionner un ou plusieurs bus
        \item Ajouter un bus
        \item Supprimer un ou plusieurs bus
        \item Voir plus d'informations sur un bus
        \item Modifier les informations d'un bus
        \item Filtrer par attributs
    \end{itemize}\\
    &
    \textbf{Restauration - Calendrier}
    \begin{itemize}
        \item Voir les menus détaillé de tout le mois par semaine
        \item Ajouter un menu
        \item Modifier un menu
        \item Supprimer un menu
    \end{itemize}\\
    Administrateur &
    \textbf{Restauration - Restaurants}
    \begin{itemize}
        \item Consulter la liste des restaurants
        \item Ajouter un restaurant
        \item Supprimer un restaurant
        \item Voir plus d'informations sur un restaurant
        \item Modifier les informations d'un restaurant
        \item Rechercher un restaurant
        \item Filtrer par les restaurants par établissements ( appartiennent à un campus ou une résidence )
    \end{itemize}\\
    &
    \textbf{Restauration - Plats \& Desserts}
    \begin{itemize}
        \item Consulter la liste des plats \& desserts
        \item Ajouter un plat/dessert
        \item Supprimer un plat/dessert
        \item Voir plus d'informations sur un plat/dessert
        \item Modifier les informations d'un plat/dessert
        \item Rechercher un plat/dessert
        \item Filtrer par plats seulement desserts seulement ou par plats \& desserts
    \end{itemize}\\
    &
    \textbf{Restauration - Ingrédients}
    \begin{itemize}
        \item Consulter la liste des ingrédients
        \item Sélectionner un ou plusieurs ingrédient(s)
        \item Ajouter un ingrédient
    \end{itemize}\\
    Administrateur&    
    \begin{itemize}
        \item Supprimer un ou plusieurs ingrédient(s)
        \item Voir plus d'informations sur un ingrédient
        \item Modifier les informations d'un ingrédient
        \item Filtrer par attribut
    \end{itemize}
\end{longtable}


\subsubsection{Besoins non fonctionnels}
Ça représente des exigences qui ne concernent pas le comportement du système. Elles identifient les contraintes internes et externes du système. Dans notre cas l’application devra respecter les exigences suivantes :\\

\noindent \textbf{Performance:} temps de réponse petit.\\
\textbf{Maintenabilité:} apporter des corrections facilement.\\
\textbf{Fiabilité:} précise et correcte.\\
\textbf{Intégrabilité:} intégrer de nouvelles fonctionnalités facilement.\\
\textbf{Portabilité:} elle peut fonctionner sur plusieurs plateformes.\\
\textbf{Disponibilité:} réaliser une fonction requise à tout moment.\\
\textbf{Sécurité:} personnaliser les accès selon l’utilisateur.\\


\subsection{Analyse des besoins}
\subsubsection{Le diagramme de contexte}
\subsubsection{Le diagramme de cas d'utilisation}
\subsubsection{Les diagrammes de séquence}


\section{Conception de la base de données}
\subsection{Le diagramme de classe}
\subsection{Passage du digramme de classe au modèle relationnel}
\subsection{Modèle relationnel}

\section{Conclusion}
		\chapter{Réalisation}

\section{Introduction}
    Dans ce chapitre nous allons présenter l'environnement de développement de l'application.\\

    Nous commencerons par les techniques utilisés puis passerons vers les bibliothèques et les frameworks qui ont aidé dans cette réalisation. Par la suite, nous présenterons les outils utilisés tout le long du processus de création.\\

    Finalement, nous présenterons quelles ques interfaces de l'application.\\

\section{Présentation des technologies utilisées}
    Le but du projet est la création d'une application full-stack web. pour cela, plusieurs outils peuvent être utilisés, parmi ces outils, nous avons choisi \acs{PERN} qui est une pile de technologies conçues justement pour la création d'un environnement de développement full-stack web. \acs{PERN}, par ses initiales, se compose de PostgreSQL, ExpressJS, React et NodeJS.\\

    \acs{PERN} est un substitué de \acs{MERN}, qui est lui-même composer de MongoDB, ExpressJS, React et NodeJS. Comme \acs{MERN}, \acs{PERN} donne la possibilité de créer des applications web full-stack avec des opérations \acs{CRUD} (Create, Read, Update, Delete). Mais \acs{PERN} utilise PostgreSQL au lieu de MongoDB est nous offre un grand support pour les fonctionnalités NoSQL, avec une forte conformité aux normes et prend en compte les transactions.\\

    \subsection{PostgreSQL\cite{postgres}}
    \begin{figure}[H]
        \centering
        \includegraphics[scale=0.2]{ACR/postgresql.png}
        \caption{Logo de PostgreSQL}
    \end{figure}
    
    PostgreSQL est système de gestion de base de données relationnel orienté objet puissant et open-source, qui utilise SQL et prend en charge en toute sécurité les charges de travil complexes en regroupant plusieurs fonctionnalités qui donnent priorité à l'extensibilité et la conformité.\\

    L'origine de PostgreSQL remonte a la base de données Ingres développer à l'université de la Californie de Berkley par Michael Stonebraker. Au années 1986, son créateur a repris le projet de zero est a décidé de le nommée POSTGRES, comme pour dire post-ingres. Ce n'est qu'en 1995 que son créateur à décidé d'ajouter les fonctionnalitées SQL est a été renommée Postgre95, et ce fut qu'à la fin des années 1996 qu'il a été renommée en PostgreSQL.\\
    
    Avec plus de 30 années de développement, PostgreSQL a gagné une forte réputation grace a son architecture, sa robustesse, son extensibilité et le dévouement des contributeurs de la communauté open-source.\\

    \subsection{ExpressJS\cite{expressjs}}
    \begin{figure}[H]
        \centering
        \includegraphics[scale=0.16]{ACR/ExpressJS-logo.png}
        \caption{Logo de ExpressJS}
    \end{figure}
    
    ExpressJS est un framework NodeJS qui fournis des fonctionnalitées robustes pour les application web et mobile, il est très minimaliste, très léger et très fléxible. Il apporte peu de surcouche et garde un coté optimiale et une execution rapide.

    \subsection{React}
    \subsection{NodeJS}

\section{Bibliothèques et Framework utilisés}
    \subsection{Axios}
    \subsection{Redux}
    \subsection{Jwt}
    \subsection{Material UI}
    \subsection{rrule}

\section{Présentation des outils utilisés}
    \subsection{Visual Studio Code}
    \subsection{Dbeaver}
    \subsection{Github}
    \subsection{Discord}
    \subsection{Lucidchart}

\section{Présentation des interfaces}
    \subsection{Interface d'accueil}
    \subsection{Interface d'authentification}
    \subsection{Interface 'kadha wakadha'}
    \subsection{Interface 'kadha wakadha'}
    \subsection{Interface 'kadha wakadha'}
    \subsection{Interface 'kadha wakadha'}
    \subsection{Interface 'kadha wakadha'}
    \subsection{Interface 'kadha wakadha'}

\newpage

\leftskip=0cm
\renewcommand{\bibname}{Référence bibliographique et webographique du chapitre 4}
\bibliographystyle{ieeetr}	
\bibliography{ACR/acr}

	\part{Conclusion Générale}
		\renewcommand{\thechapter}{}
\renewcommand{\chaptername}{}
\chaptermark{Conclusion}

L'objectif de notre projet etait de concevoir une plateforme uniforme pour les directions des œuvres universitaires qui permettrait d'un coté a cette diréction un suivi en temps réel des ressources a leurs dispositions et une performance et des économies accrus grâce a un meilleur vue d'ensembles des mouvement de ces ressources, d'un autre coté aux étudiants de pouvoir accéder a plusieurs bout d'informations pertinante toujours au même et unique endroit.\\

Tout au long de ce mémoire nous avons présenté les différentes phases de réalisation de notre projet. Nous avons commencer par définir le progiciel de gestion interne aussi connu en tant que \acs{ERP}, nous avons parcouru son historique, et vue les avantages et les inconvenients de ce derniers. Ensuite, nous avons décrit les besoins des futurs utilisateurs en utilisant UML. Ceci a conduit à une analyse plus approfondie puis à la conception des fonctions applicatives que nous avons mises en œuvre dans le processus de production.\\

Cette dernière est basée sur des techniques fiables, puissantes et évolutives.Le front-end est géré par React qui est une bibliothèque Javascript très puissante. le back-end est construit a l'aide d'ExpressJS qui est une bibliothèque Javascript lui aussi qui est très versatile et très robuste. La base de données a été créée avec PostgreSQL qui est un gestionnaire de base de données qui utilise \acs{SQL} mais qui le complémente de manière a favorisé l'extensibilité et la conformité. Tout ceci viens s'emboiter de façons à ce que les éventuelles fonctionnalités qui viendront s'ajouter à celles déjà implémenté, tel que la gestion des ingrédients, la gestion des restaurants, la gestion des calendriers des menus et des transports, la gestion des dossiers d'hébergements et des dossiers bourse et bien d'autres, peuvent être ajoutées avec une grande simplicité.\\

Malgré les inconvénients et les limites imposées par certaines contraintes, qui ont créé plusieurs obstacles a l'amélioration de l'application. Réaliser une interface simple, intuitive et ergonomique et des fonctionnalités pratiques reste complexes en vue du besoin nécessaire d'un feedback régulier et directe de la part des futurs utilisateurs de cette application.\\

Cependant ce projet n'est qu'à son début et à beaucoup de potentiel d'amélioration avec l'intégration d'autres modules pour couvrir l'ensemble des besoins de la direction des œuvres universitaires. Ainsi que l'ajout et la modification de quelles ques fonctionnalités telles qu'un tableau de bord pour voir toutes les données pertinente dans une seule interface et la possibilité de changer de langues.\\

Commencer un projet à partir de zéro était une opportunité que vous devaient saisir. Ce projet nous a permis d'évoluer, d'approfondir et d'acquérir de nouvelles connaissances. Nous avons découvert et maitriser plusieurs outils, technologies et framework. Nous sommes donc fières de notre travail et nous sommes sur de pouvoir relever notre prochain défi.\\


	\leftskip=0cm
	\renewcommand{\bibname}{Référence bibliographique et webographique}
	\bibliographystyle{ieeetr}
	\bibliography{ERP/erp,DOU/dou,ACR/acr}
	
\end{document}