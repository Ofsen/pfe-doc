\chapter{Les Progiciels de Gestion Internes}

\section{Introduction}
Le but de ce chapitre est de présenter globalement le progiciel de gestion interne aussi appeler \acs{ERP}.\\

Dans un premier temps, nous définirons le concept d'\acs{ERP} et son évolution dans le temps.\\

Par la suite, nous aborderons les avantages liés a l'intégration d'un tel système dans les deux aspects administratif et opérationnel ainsi que ses inconvénients.\\

Nous finirons avec les multiples fonctionnalités de l'\acs{ERP}.\\

\section{Définition}
L'acronyme \acs{ERP} signifie Entreprise Resource Planning\cite{def-erp}, et sa similitude en français est Progiciel de Gestion Intégré abrévié \acs{PGI}.\\

Contrairement au \acs{MRP} (Manufacturing Resource Planning) qui se contente de la planification des besoins, l'\acs{ERP} est un logiciel qui permet la gestion de tous les sous-systèmes de l'entreprise et la coordination de ces sous-systèmes.\\

Pour y parvenir l’\acs{ERP} intègre toutes les fonctions utiles de l'entreprise sous forme de modules qui partagent une seule base de données, ce qui permet l'échange d'informations entre modules, dans ce cas, on parle de moteurs de workflow\footnote{"the sequence of steps involved in moving from the beginning to the end of a working process" - \href{https://www.merriam-webster.com/dictionary/workflow}{merriam-webster.com/dictionary - Definition of workflow}}.\\

\section{Historique}
Joseph Orlicky a été à l'origine de la création de l'\acs{ERP}.\cite{hist-erp} Il a créé l'acronyme de \acs{MRP} dans les années 1960, qui est l'ancêtre de la planification de la demande matérielle d'\acs{ERP}. \acs{MRP} répond principalement aux besoins de planification de l'entreprise.\\

Le concept d'\acs{ERP} tel que nous le connaissons est apparu pour la première fois dans les années 1990, mais avec l'avènement d'internet, il n'a commencé à se développer que dans les années 2000. L'utilisation de l'\acs{ERP} s'est généralisée et a évolué vers l'\acs{ERP} tel que nous le connaissons aujourd'hui.

\section{Avantages liés à l’intégration d’un \acs{ERP}}
Plusieurs études ont démontré les bénéfices de la mise en place d'un \acs{ERP}, dont l'une a été menée par Aberdeen Group,\cite{avantages} qui a quantifié et publié les résultats suivants :\\

\begin{itemize}
    \item Réduction des coûts d’opérations de 22\%
    \item Réduction des coûts d’administration de 20\%
    \item Réduction d’inventaires de 17\%
    \item Amélioration du temps de livraison de 19\%
    \item Amélioration du respect des délais et des budgets de 17\%\\
\end{itemize}

Même les entreprises en difficulté ont réalisé des avantages en intégrant l'\acs{ERP}, et le résultat est :\\

\begin{itemize}
    \item Réduction des coûts d’opérations de 7\%
    \item Réduction des coûts d’administration de 4\%
    \item Réduction d’inventaires de 9\%
    \item Amélioration du temps de livraison de 11\%
    \item Amélioration du respect des délais et des budgets de 6\%\\
\end{itemize}

Comme le souligne la recherche, les avantages en pourcentage ne semblent pas impressionnants, mais pour chaque million de dollars dépensé en coûts d'exploitation, des économies de 70 000 \$ sont réalisées.\\

En effet, on peut constater l'amélioration de la productivité et de la maturité des entreprises. Pour y parvenir, l'\acs{ERP} a été amélioré sous plusieurs aspects\cite{aspects} : \\

\subsection{Aspect administratif}
En consolidant tous les systèmes de l'entreprise en une seule application, l'installation de l'\acs{ERP} peut réduire les coûts d'exploitation et de maintenance, et parce que l'\acs{ERP} a une architecture modulaire, il fournit une infrastructure qui peut assurer la flexibilité à l'avenir lorsque des changements se produisent.\\

Une seule application, donc une seule base de données, cette seule base de données permet de gagner du temps. Réduire la quantité d'informations inutiles et évitez les saisies multiples. L'installation de l'\acs{ERP} résout le problème des informations incohérentes et fiabilise les données enregistrées.\\

De plus les activités manuelles de traitement, de comparaison et de recherche réalisée par les employés dans le cadre de l'interface des différents services sont évitées. Cela conduit à un gain de croissance, de temps et de productivité administrative.\\ 

\subsection{Aspect opérationnel}
L'utilisation de l'\acs{ERP} permet d'éliminer les risques opérationnels et les risques de pertes liés aux erreurs humaines ou aux défaillances du contrôle interne, et les fraudes qui peuvent être provoquées par les défaillances du système d'information existant. Les coûts supplémentaires inutiles dus aux dysfonctionnements sont réduits et une pertinence des informations partagées est gagnée.\\

L'\acs{ERP} permet également un suivi au niveau de l'achat jusqu'à la vente. En effet, dès la création de la commande, des données telles que la marge et le crédit sont générés automatiquement de manière dynamique pour réaliser l'intégration financière. Avec cette fonction, l'\acs{ERP} aide les managers dans le processus de planification et de prise de décision, et leur permet d'améliorer la gestion des ressources, ainsi améliorer la prise de décision opérationnelle.\\

De plus, les services de finances bénéficient de la centralisation. Cette centralisation permet de réunir les taches dans un seul endroit, ce qui a son tour permet l'amélioration de la productivité en réduisant le nombre d'employés nécessaires qui travaille sur la même tâche, cela permet d'augmenter les économies d’échelles notamment en matière de facturation.\\

\section{Inconvénients}
L'\acs{ERP} offre des avantages importants, mais une telle solution doit présenter certains inconvénients.\\

Les projets \acs{ERP} entraînent généralement des coûts lors de la configuration et de la maintenance. De plus, la complexité des programmes utilisés nécessite l'utilisation et la maintenance de serveurs puissants. Cela signifie que comme le montre l'étude CXP 2017\cite{cxp-2017}, les coûts sont souvent dépassés.\\

\begin{figure}[H]
    \centering
    \includegraphics[scale=0.4]{ERP/graph-depassement-budget.jpg}
    \caption{Taux du dépassement de budget lors de l’implémentation d’un ERP}
\end{figure} 

On peut constater qu’en 2017 plus de 60\% des entreprises qui ont implémenté un \acs{ERP} ont dépassé le budget prévu, 58\% en 2016 et 55\% en 2015.\\

En plus du coût, comme le montre l'étude 2010 du rapport \acs{ERP} du cabinet de conseil Panorama Consulting\cite{panorama-consulting}, un projet d'une telle envergure peut nécessiter plus de temps et de ressources que prévu.\\

\begin{figure}[H]
    \centering
    \includegraphics[scale=0.65]{ERP/graph-taux-implementation.png}
    \caption{Taux des dépassements des délais lors de l’implémentation d’un ERP}
\end{figure} 

Cette étude montre que plus de 35,5\% des entreprises ont mis en place un \acs{ERP} et constatent que le délai de mise en place a dépassé le délai autorisé. Il faut également noter que la durée moyenne de mise en place de l'\acs{ERP} est de 18 ou 4 mois, d'un éditeur à l'autre.\\

\section{Fonctionnalités}
Un \acs{ERP} est donc composé de modules\cite{modules}, ces modules son interconnecté et coexistent dans le système centralisé. Ce système centralisé est connecté à une seule base de données, ce qui donne l'avantage d'avoir tout un système et toutes les données requissent dans un seul endroit.\\

\begin{figure}[H]
    \centering
    \includegraphics[scale=0.3]{ERP/erp-modules.png}
    \caption{Les modules d'un ERP}
\end{figure} 

Les besoins de chaque entreprise varient considérablement l'une de l'autre, mais l'\acs{ERP} donnent à celles-ci le pouvoir de choisir les modules dont elles ont vraiment besoin. Une entreprise orientée produite peut vouloir centraliser son inventaire et donc implémenter un module de gestion de stock, tandis qu'une entreprise orientée services pourront envisager l'implémentation d'un module de gestion de la relation clients.\\

Malgré leurs différences, ces deux types d'entreprise auront surement besoin de coordonner et gérer leurs employés ce qui les pousserait à envisager l'implementer d'un module de gestion des ressources humaine.\\

L'\acs{ERP} gère et organise automatiquement et dynamiquement les informations des différents services de l'entreprise, de l'approvisionnement des ressources aux ventes en passant par la production, ces fonctions\cite{funcs} sont nombreuses. Les modules les plus couramment utilisés sont :\\

\begin{itemize}
    \item Gestion de production
    \item Gestion de stock et d’inventaire
    \item Gestion des ressources humaines 
    \item Gestion de projet
    \item Gestion comptabilité
    \item Gestion commerciale
    \item Gestion d’achats
    \item CRM : Gestion des relations clients\\
\end{itemize}

Chaque module couvre ses propres fonctions, et le tableau suivant résume certains des modules et les fonctions qu'ils fournissent.

\begin{table}[H]
    \begin{center}
        
        \begin{tabular}{|F{4cm}|R{10cm}|}
            \hline
            \textbf{Modules}  & \makecell[c]{\textbf{Fonctionnalités}} \\
            \hline
            Achats
            &
            Gestion de toutes les transactions comptables, telle que les bons de commande pour l’approvisionnement. Etc.\\
            
            \hline
            Stock
            &
            Gestion des mouvements du stock, état du stock, entreposage.\\
            
            \hline
            Production
            &
            La gestion de la production, permet de réguler l’offre et les besoins en
            ressources par apport à la demande, impliquent la planification des ordres
            de fabrication et le contrôle de qualité.\\
            
            \hline
            Gestion de projet
            &
            Gestion de l’ensemble des projets de l’entreprise, de ces tâches et de ces plannings.\\
            
            \hline
            Ressources humaines
            &
            Gestion des ressources humaines et l’organisation de la rémunération des employés ainsi que des plannings de travail de ceux-ci.\\
            
            
            \hline
            Comptabilité
            &
            Gestion des obligations comptable auxquelles l’entreprise est soumise et suivie en temps réel de la santé financière de celle-ci, ainsi que de la gestion de facturation et des multidevises.\\
            
            \hline
            Commerciale
            &
            Gestion de l’aspect commerciale de l’entreprise, permet la gestion de l’ensemble des commandes clients et de leur facturation, permet aussi la réalisation de devis rapide et précise.\\
            
            \hline
            CRM
            &
            Gestion des relations clients, permet de réaliser de meilleurs suivis de
            l’environnement : clients, fournisseurs, prospects. etc.\\
            
            
            \hline
        \end{tabular}	
        \caption{Les Modules d'un \acs{ERP} et leurs fonctionnalités}
    \end{center}
\end{table}

\section{le marché des \acs{ERP}s}
Les \acs{ERP} en marché peuvent etre divisé en deux catégories: les \acs{ERP} open source et les \acs{ERP} propriétaires.

    \subsection{Les \acs{ERP} open source}
    Les \acs{ERP} open source sont ceux qui ont leurs code source est libre d'acces. Il est donc téléchargeable gratuitement et modifiable pour ajouter les fonctions nécessaires pour répondre aux besoins d'une entreprise ou la suppression des modules non nécessaire ou inutilisable déja existant.\\

    L'un des principaux leaders des \acs{ERP} open sources est nommée Odoo:

    \begin{figure}[H]
        \centering
        \includegraphics[scale=0.1]{ERP/Odoo_logo.png}
        \caption{Logo Odoo}
    \end{figure} 

    Odoo\cite{odoo} est l'un des principaux logiciels \acs{ERP} open source, également connu sous le nom d'OpenERP\footnote{Open source \acs{ERP} - \acs{ERP} libre}. Il est composé de diverses applications et modules tels que le CRM, les ventes, la fabrication, la gestion de projet, les achats et la gestion des ressources humaines.\\

    Le développement et la mise en œuvre d'Odoo vous permet de choisir parmi des milliers de modules disponibles dans la boutique. Le logiciels \acs{ERP} Odoo fournit des services de bout en bout tels que la personnalisation, la mise en œuvre, l'intégration et l'assistance à la formation.\\

    \subsection{Les \acs{ERP} propriétaires}
    Les \acs{ERP} propriétaires sont les \acs{ERP} conçus par des entreprises qui ont une grande expertise dans la conception de logiciels et de systèmes informatiques. Comme tout \acs{ERP}, il contient différents modules qui répondent aux besoins des entreprises sauf que le système est fourni avec une licence payante, des limitations d'utilisation et des obligations et responsabilités dont le client doit suivre.\\
    
    L'avantage de ce type d'\acs{ERP} est de pouvoir profiter d'un savoir-faire reconnu, d'un accompagnement pendant toutes les étapes du projet d'ERP, d'un service personnalisé et dédié assurant la maintenance et le service après-vente.\\

    Parmis les leaders des \acs{ERP} propriétaires on peut citer SAP\cite{sap}, Microsoft Dynamics\cite{ms-dynamics} ou même Oracle \acs{ERP} Cloud\cite{oracle}.

    \begin{figure}[H]
        \centering
        \includegraphics[scale=0.03]{ERP/SAP_logo.png}
        \includegraphics[scale=0.3]{ERP/MS-Dynamics-Logopng.png}
        \includegraphics[scale=0.2]{ERP/Oracle_Cloud_logo.jpg}
        \caption{Logo SAP, Microsoft Dynamics \& Oracle \acs{ERP} Cloud}
    \end{figure}

    Chacun de ces leaders propose sa propre solution que nous allons décrire dans le tableau suivant:

    \begin{table}[H]
        \begin{center}
            
            \begin{tabular}{|F{2.3cm}|R{3.5cm}|R{3.5cm}|R{3.5cm}|}
                \hline
                \textbf{ } & SAP & Microsoft & Oracle \\
                \hline
                Cible & \acs{PME}, \acs{ETI}, Grandes entreprises & \acs{PME}, \acs{ETI}, Grandes entreprises & \acs{PME}, \acs{ETI}, Grandes entreprises\\
                
                \hline
                Solution & SAP ERP, S/4 HANNA, S/4 Cloud, SAP Business One, SAP Business By Design & Oracle \acs{ERP} Cloud & Microsoft Dynamics 365\\
                
                \hline
                Modules
                &
                23 modules compris dans  3 catégories: Logistique, Comptabilité et RH
                &
                \begin{itemize}
                    \item Ventes
                    \item Marketing
                    \item Service
                    \item Finances
                    \item Opérations
                    \item Commerce
                    \item RH
                \end{itemize}
                &
                \begin{itemize}
                    \item Finances
                    \item Gestion de projets
                    \item Gestion de l'approvisionnement
                    \item Gestion des risques et conformité
                    \item Gestion des performances d'entreprise
                    \item Gestion de la chaîne d'approvisionnement et fabrication
                    \item Analytique 
                \end{itemize}
                \\

                \hline
                SGBD
                &
                SAP HANNA, Microsoft SQL Server, Oracle, My
                SQL.
                &
                Oracle Cloud
                &
                SQL Server
                \\
                
                \hline
            \end{tabular}	
            \caption{Description des solution \acs{ERP} propriétaires: SAP, Microsoft Dynamics et Oracle \acs{ERP} Cloud}
        \end{center}
    \end{table}    

\section{Conclusion}
Nous avons présenté l'\acs{ERP}, sa définition, son histoire au cours des années en plus des avantages, que ce sois dans l'aspect administratif ou opérationnel, et des inconvénients qu'il peut apporter.\\

Nous avons aussi montré qu'un \acs{ERP} peut être construit de plusieurs manières différentes et cela grâce à ses multiples fonctionnalités modulaires.\\

Dans le chapitre suivant, nous allons approfondir nos recherches et étudier le concept d'entreprise et la direction des œuvres universitaires.\\