\chapter{Réalisation}

\section{Introduction}
    Dans ce chapitre nous allons présenter l'environnement de développement de l'application.\\

    Nous commencerons par les techniques utilisés puis passerons vers les bibliothèques et les frameworks qui ont aidé dans cette réalisation. Par la suite, nous présenterons les outils utilisés tout le long du processus de création.\\

    Finalement, nous présenterons quelles ques interfaces de l'application.\\

\section{Présentation des technologies utilisées}
    Le but du projet est la création d'une application full-stack web. pour cela, plusieurs outils peuvent être utilisés, parmi ces outils, nous avons choisi \acs{PERN} qui est une pile de technologies conçues justement pour la création d'un environnement de développement full-stack web. \acs{PERN}, par ses initiales, se compose de PostgreSQL, ExpressJS, React et NodeJS.\\

    \acs{PERN} est un substitué de \acs{MERN}, qui est lui-même composer de MongoDB, ExpressJS, React et NodeJS. Comme \acs{MERN}, \acs{PERN} donne la possibilité de créer des applications web full-stack avec des opérations \acs{CRUD} (Create, Read, Update, Delete). Mais \acs{PERN} utilise PostgreSQL au lieu de MongoDB est nous offre un grand support pour les fonctionnalités NoSQL, avec une forte conformité aux normes et prend en compte les transactions.\\

    \subsection{PostgreSQL\cite{postgres}}
    \begin{figure}[H]
        \centering
        \includegraphics[scale=0.2]{ACR/postgresql.png}
        \caption{Logo de PostgreSQL}
    \end{figure}
    
    PostgreSQL est système de gestion de base de données relationnel orienté objet puissant et open-source, qui utilise SQL et prend en charge en toute sécurité les charges de travil complexes en regroupant plusieurs fonctionnalités qui donnent priorité à l'extensibilité et la conformité.\\

    L'origine de PostgreSQL remonte a la base de données Ingres développer à l'université de la Californie de Berkley par Michael Stonebraker. Au années 1986, son créateur a repris le projet de zero est a décidé de le nommée POSTGRES, comme pour dire post-ingres. Ce n'est qu'en 1995 que son créateur à décidé d'ajouter les fonctionnalitées SQL est a été renommée Postgre95, et ce fut qu'à la fin des années 1996 qu'il a été renommée en PostgreSQL.\\
    
    Avec plus de 30 années de développement, PostgreSQL a gagné une forte réputation grace a son architecture, sa robustesse, son extensibilité et le dévouement des contributeurs de la communauté open-source.\\

    \subsection{ExpressJS\cite{expressjs}}
    \begin{figure}[H]
        \centering
        \includegraphics[scale=0.16]{ACR/ExpressJS-logo.png}
        \caption{Logo de ExpressJS}
    \end{figure}
    
    ExpressJS est un framework NodeJS qui fournis des fonctionnalitées robustes pour les application web et mobile, il est très minimaliste, très léger et très fléxible. Il apporte peu de surcouche et garde un coté optimiale et une execution rapide.

    \subsection{React}
    \subsection{NodeJS}

\section{Bibliothèques et Framework utilisés}
    \subsection{Axios}
    \subsection{Redux}
    \subsection{Jwt}
    \subsection{Material UI}
    \subsection{rrule}

\section{Présentation des outils utilisés}
    \subsection{Visual Studio Code}
    \subsection{Dbeaver}
    \subsection{Github}
    \subsection{Discord}
    \subsection{Lucidchart}

\section{Présentation des interfaces}
    \subsection{Interface d'accueil}
    \subsection{Interface d'authentification}
    \subsection{Interface 'kadha wakadha'}
    \subsection{Interface 'kadha wakadha'}
    \subsection{Interface 'kadha wakadha'}
    \subsection{Interface 'kadha wakadha'}
    \subsection{Interface 'kadha wakadha'}
    \subsection{Interface 'kadha wakadha'}

\newpage

\leftskip=0cm
\renewcommand{\bibname}{Référence bibliographique et webographique du chapitre 4}
\bibliographystyle{ieeetr}	
\bibliography{ACR/acr}