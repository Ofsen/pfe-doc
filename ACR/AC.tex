\chapter{Analyse \& Conception}

\section{Introduction}
Le but de ce chapitre est d'aborder les concepts essentiels a la réalisation du projet.\\

En première partie, nous ferons un rappel sur les généralités d'UML, sa définition et ses diagrammes.\\

Par la suite, nous passerons à l'analyse des principaux utilisateurs du système et les besoins de ces derniers.\\

Nous terminerons par une conception des diagrammes de classes et des diagrammes de séquences ainsi qu'une présentation de la structure de la base de données.\\

\section{Présentation d'\acs{UML}}
    \acs{UML} est un acronym qui signifie Unified Modeling Langage ou Langage de Modélisation Unifié en français. En simple, \acs{UML} est une approche moderne et méthodique de la modélisation et de la documentation des applications. C'est l'une des techniques de modélisation des processus d'entreprise les plus populaires.\\
    
    Son approche est basée sur des représentations schématiques des composants d'un logiciel. Cette représentations visuelles nous donnent la possibilité de mieux mesurer les éventuelles erreurs ou défauts des logiciels.\\

    \subsection{Les diagrammes \acs{UML}}
        \subsubsection{Diagrammes des cas d'utilisation}
        \subsubsection{Diagrammes de classes}
        \subsubsection{Diagrammes de séquence}

\section{Spécification et analyse des besoins}
    \subsection{Spécification des besoins}
        \subsubsection{Besoins fonctionnels}
        \subsubsection{Besoins non fonctionnels}

    \subsection{Analyse des besoins}
        \subsubsection{Le diagramme de contexte}
        \subsubsection{Le diagramme de cas d'utilisation}
        \subsubsection{Les diagrammes de séquence}
        
        
\section{Conception de la base de données}
    \subsection{Le diagramme de classe}
    \subsection{Passage du digramme de classe au modèle relationnel}
    \subsection{Modèle relationnel}

\section{Conclusion}