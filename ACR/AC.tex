\chapter{Analyse \& Conception}

\section{Introduction}
Le but de ce chapitre est d'aborder les concepts essentiels a la réalisation du projet.\\

En première partie, nous ferons un rappel sur les généralités d'UML, sa définition et ses diagrammes.\\

Par la suite, nous passerons à l'analyse des principaux utilisateurs du système et les besoins de ces derniers.\\

Nous terminerons par une conception des diagrammes de classes et des diagrammes de séquences ainsi qu'une présentation de la structure de la base de données.\\

\section{Présentation d'\acs{UML}}
\acs{UML} est un acronym qui signifie Unified Modeling Langage ou Langage de Modélisation Unifié en français, est apparu pour la première fois dans les années 1990. En simple, \acs{UML} est une approche moderne et méthodique de la modélisation et de la documentation des applications. C'est l'une des techniques de modélisation des processus d'entreprise les plus populaires.\\

Son approche est basée sur des représentations schématiques des composants d'un logiciel. Cette représentations visuelles nous donnent la possibilité de mieux mesurer les éventuelles erreurs ou défauts des logiciels.\\


\subsection{Les diagrammes \acs{UML}}
\subsubsection{Diagrammes des cas d'utilisation}
\subsubsection{Diagrammes de classes}
\subsubsection{Diagrammes de séquence}

\section{Analyse}
La phase d’analyse débute par la spécification des besoins fonctionnels du système, ensuite définir les acteurs ainsi que leurs différentes tâches. Ceci nous mène vers un résultat qui est un diagramme de cas d’utilisation traduisant la dynamique système qui sera utilisé dans la phase de conception.

\subsection{Spécification des besoins}
\subsubsection{Besoins fonctionnels}
C’est une description des caractéristiques du système et de ses fonctionnalités. En partant de ce principe notre application doit permettre :

\begin{longtable}{|Z{3cm}|L{10cm}|}
    \caption{Les acteurs et leurs fonctions respectives}\\
    \hline
    \textbf{Acteur} & \textbf{Fonctions} \\
    \hline
    \endfirsthead
    \multicolumn{2}{c}%
    {\tablename\ \thetable\ -- \textit{Suite de la page précedente}} \\
    \hline
    \textbf{Acteur} & \textbf{Fonctions} \\
    \hline
    \endhead
    \hline \multicolumn{2}{r}{\textit{Suite dans la page suivante}} \\
    \endfoot
    \hline
    \endlastfoot
    Visiteurs &
        \textbf{Consultation de la page d’accueil}
        \begin{itemize}
            \item Consulter le service d'hébergements
            \item Télécharger le document d'aide à la constitution du dossier d'hébergement
            \item Consulter le service de restauration
            \item Consulter le service des transports
            \item Consulter le service des bourses
            \item Télécharger le document d'aide à la constitution du dossier de bourse
            \item Télécharger le document d'aide au renouvellement du dossier de bourse
            \item Télécharger le document d'aide au transfert du dossier de bourse
        \end{itemize} \\
        &
        \textbf{Consultation du calendrier des menus de chaque restaurant d'un mois défini}
        \begin{itemize}
            \item Changer le mois du calendrier des menus
            \item Changer l'affichage des menus par mois, semaine, jour de travail, jour et par agenda
            \item Voir plus d'informations sur un seul menu
        \end{itemize} \\
        &
        \textbf{Consultation du calendrier de tous les trajets de chaque Campus/Résidence d'un mois défini}
        \begin{itemize}
            \item Changer le mois du calendrier des trajets
            \item Changer l’affichage des trajets par mois, semaine, jour de travail, jour et par agenda
            \item Voir plus de détails sur un seul trajet
        \end{itemize}
     \\
     \hline
    & 
    En plus des fonctionnalités du visiteur il peut aussi:\\
    Administrateur &
    \textbf{Hébergement}
    \begin{itemize}
        \item Consulter la liste des dossiers d'hébergement
        \item Sélectionner un ou plusieurs dossiers d'hébergement
        \item Ajouter un dossier d'hébergement
        \item Supprimer un ou plusieurs dossier(s) d'hébergement
    \end{itemize}\\
    &
    \begin{itemize}
        \item Voir plus d'informations sur un dossier d'hébergement
        \item Valider un dossier d'hébergement
        \item Refuser un dossier d'hébergement
    \end{itemize}\\
    &
    \textbf{Bourse}
    \begin{itemize}
        \item Consulter la liste des dossiers de bourse
        \item Sélectionner un ou plusieurs dossiers de bourse
        \item Ajouter un dossier de bourse
        \item Supprimer un ou plusieurs dossier(s) de bourse
        \item Voir plus d'informations sur un dossier de bourse
        \item Valider un dossier de bourse
        \item Refuser un dossier de bourse
        \item Changer l'ordre d'affichage par attributs
    \end{itemize}\\
    Administrateur &
    \textbf{Utilisateurs}
    \begin{itemize}
        \item Consulter la liste des utilisateurs
        \item Sélectionner un ou plusieurs utilisateur(s)
        \item Ajouter un utilisateur
        \item Supprimer un ou plusieurs utilisateur(s)
        \item Voir plus d'informations sur un utilisateur
        \item Modifier les informations d'un utilisateur
        \item Filtrer par attributs
    \end{itemize}\\
    &
    \textbf{Campus \& Résidences}
    \begin{itemize}
        \item Consulter la liste des Campus \& Résidences
        \item Ajouter un campus ou une résidence
        \item Supprimer un campus ou une résidence
        \item Voir plus d'informations sur un campus ou une résidence
        \item Modifier les informations d'un campus ou une résidence
        \item Rechercher un campus ou une résidence
        \item Filtrer par campus seulement résidences seulement ou par campus \& résidences
    \end{itemize}\\
    &
    \textbf{Transport - Calendrier}
    \begin{itemize}
        \item Voir les trajets détaillé de tout le mois
        \item Ajouter un trajet
        \item Modifier un trajet
        \item Supprimer un trajet
    \end{itemize}\\
    &
    \textbf{Transport - Bus}
    \begin{itemize}
        \item Consulter la liste des bus
        \item Sélectionner un ou plusieurs bus
        \item Ajouter un bus
        \item Supprimer un ou plusieurs bus
        \item Voir plus d'informations sur un bus
        \item Modifier les informations d'un bus
        \item Filtrer par attributs
    \end{itemize}\\
    &
    \textbf{Restauration - Calendrier}
    \begin{itemize}
        \item Voir les menus détaillé de tout le mois par semaine
        \item Ajouter un menu
        \item Modifier un menu
        \item Supprimer un menu
    \end{itemize}\\
    Administrateur &
    \textbf{Restauration - Restaurants}
    \begin{itemize}
        \item Consulter la liste des restaurants
        \item Ajouter un restaurant
        \item Supprimer un restaurant
        \item Voir plus d'informations sur un restaurant
        \item Modifier les informations d'un restaurant
        \item Rechercher un restaurant
        \item Filtrer par les restaurants par établissements ( appartiennent à un campus ou une résidence )
    \end{itemize}\\
    &
    \textbf{Restauration - Plats \& Desserts}
    \begin{itemize}
        \item Consulter la liste des plats \& desserts
        \item Ajouter un plat/dessert
        \item Supprimer un plat/dessert
        \item Voir plus d'informations sur un plat/dessert
        \item Modifier les informations d'un plat/dessert
        \item Rechercher un plat/dessert
        \item Filtrer par plats seulement desserts seulement ou par plats \& desserts
    \end{itemize}\\
    &
    \textbf{Restauration - Ingrédients}
    \begin{itemize}
        \item Consulter la liste des ingrédients
        \item Sélectionner un ou plusieurs ingrédient(s)
        \item Ajouter un ingrédient
    \end{itemize}\\
    Administrateur&    
    \begin{itemize}
        \item Supprimer un ou plusieurs ingrédient(s)
        \item Voir plus d'informations sur un ingrédient
        \item Modifier les informations d'un ingrédient
        \item Filtrer par attribut
    \end{itemize}
\end{longtable}


\subsubsection{Besoins non fonctionnels}
Ça représente des exigences qui ne concernent pas le comportement du système. Elles identifient les contraintes internes et externes du système. Dans notre cas l’application devra respecter les exigences suivantes :\\

\noindent \textbf{Performance:} temps de réponse petit.\\
\textbf{Maintenabilité:} apporter des corrections facilement.\\
\textbf{Fiabilité:} précise et correcte.\\
\textbf{Intégrabilité:} intégrer de nouvelles fonctionnalités facilement.\\
\textbf{Portabilité:} elle peut fonctionner sur plusieurs plateformes.\\
\textbf{Disponibilité:} réaliser une fonction requise à tout moment.\\
\textbf{Sécurité:} personnaliser les accès selon l’utilisateur.\\


\subsection{Analyse des besoins}
\subsubsection{Le diagramme de contexte}
\subsubsection{Le diagramme de cas d'utilisation}
\subsubsection{Les diagrammes de séquence}


\section{Conception de la base de données}
\subsection{Le diagramme de classe}
\subsection{Passage du digramme de classe au modèle relationnel}
\subsection{Modèle relationnel}

\section{Conclusion}