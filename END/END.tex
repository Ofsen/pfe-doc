\renewcommand{\thechapter}{}
\renewcommand{\chaptername}{}
\chaptermark{Conclusion}

L'objectif de notre projet etait de concevoir une plateforme uniforme pour les directions des œuvres universitaires qui permettrait d'un coté a cette diréction un suivi en temps réel des ressources a leurs dispositions et une performance et des économies accrus grâce a un meilleur vue d'ensembles des mouvement de ces ressources, d'un autre coté aux étudiants de pouvoir accéder a plusieurs bout d'informations pertinante toujours au même et unique endroit.\\

Tout au long de ce mémoire nous avons présenté les différentes phases de réalisation de notre projet. Nous avons commencer par définir le progiciel de gestion interne aussi connu en tant que \acs{ERP}, nous avons parcouru son historique, et vue les avantages et les inconvenients de ce derniers. Ensuite, nous avons décrit les besoins des futurs utilisateurs en utilisant UML. Ceci a conduit à une analyse plus approfondie puis à la conception des fonctions applicatives que nous avons mises en œuvre dans le processus de production.\\

Cette dernière est basée sur des techniques fiables, puissantes et évolutives.Le front-end est géré par React qui est une bibliothèque Javascript très puissante. le back-end est construit a l'aide d'ExpressJS qui est une bibliothèque Javascript lui aussi qui est très versatile et très robuste. La base de données a été créée avec PostgreSQL qui est un gestionnaire de base de données qui utilise \acs{SQL} mais qui le complémente de manière a favorisé l'extensibilité et la conformité. Tout ceci viens s'emboiter de façons à ce que les éventuelles fonctionnalités qui viendront s'ajouter à celles déjà implémenté, tel que la gestion des ingrédients, la gestion des restaurants, la gestion des calen\acs{DRI} ers des menus et des transports, la gestion des dossiers d'hébergements et des dossiers bourse et bien d'autres, peuvent être ajoutées avec une grande simplicité.\\

Malgré les inconvénients et les limites imposées par certaines contraintes, qui ont créé plusieurs obstacles a l'amélioration de l'application. Réaliser une interface simple, intuitive et ergonomique et des fonctionnalités pratiques reste complexes en vue du besoin nécessaire d'un feedback régulier et directe de la part des futurs utilisateurs de cette application.\\

Cependant ce projet n'est qu'à son début et à beaucoup de potentiel d'amélioration avec l'intégration d'autres modules pour couvrir l'ensemble des besoins de la direction des œuvres universitaires. Ainsi que l'ajout et la modification de quelles ques fonctionnalités telles qu'un tableau de bord pour voir toutes les données pertinente dans une seule interface et la possibilité de changer de langues.\\

Commencer un projet à partir de zéro était une opportunité que vous devaient saisir. Ce projet nous a permis d'évoluer, d'approfondir et d'acquérir de nouvelles connaissances. Nous avons découvert et maitriser plusieurs outils, technologies et framework. Nous sommes donc fières de notre travail et nous sommes sur de pouvoir relever notre prochain défi.\\
