Actuellement, le monde connaît une avancée technologique considérable dans tous les secteurs et cela grâce à l'informatique qui est une science qui étudie les techniques du traitement automatique de l'information. Elle joue un rôle important dans le développement de l'entreprise et d'autres établissements (ex : administrations hospitalières).\\

Avant l'invention de l'ordinateur, on enregistrait toutes les informations manuellement sur des supports en papier ce qui engendrait beaucoup de problèmes tel que la perte de temps considérable dans la recherche de ces informations ou la dégradation de ces dernières ...Etc.\\

Ainsi, jusqu'à présent, l'ordinateur reste le moyen le plus sûr pour le traitement et la sauvegarde de l'information. Cette invention a permis d'informatiser les systèmes de données des entreprises, ce qui est la partie essentielle dans leur développement aujourd'hui surtout depuis l'apparition des nouvelles technologies, notamment les ERP.\\

L'ERP (Entreprise Ressource Planning) ou Progiciel de Gestion Intégré en français est une solution informatique destinée au pilotage des structures et entreprises. Son fonctionnement est basé sur le regroupement d'un ensemble de tâches liées aux activités d'une société.\\

L'ERP est capable de couvrir une large palette de gestion d'une entreprise allant de la gestion comptable à la gestion commerciale sans oublier la gestion de la paye ou encore la gestion des stocks. L'histoire des logiciels ERP a permis au fin des époques de doter cet outil d'une couverture fonctionnelle toujours plus stratégique.\\

C'est dans ce cadre que s'inscrit notre projet fin d'études qui a pour objectif de mettre en place un module de gestion des œuvres universitaires (que ce soit de la restauration, des bourses, d'hébergement ou encore des transports). Pour cela nous avons organisé notre travail en quartes chapitres :\\

\begin{itemize}
    \item Le premier chapitre intitulé \textbf{"La solution ERP"}, est dédié à la description du concept de l'entreprise et les problématiques des entreprises et les solutions qu'offre un ERP à celles-ci.
    \item Le deuxième chapitre intitulé \textbf{"Les œuvres universitaires"}, est consacré à la présentation des œuvres universitaires et l'impact des ERP sur celles-ci.
    \item Le troisième chapitre intitulé \textbf{"Analyse et Conception"}, présente les étapes de la conception et la modélisation de notre projet.
    \item Le quatrième chapitre intitulé \textbf{"Réalisation"}, dans ce chapitre nous présenterons l'environnement et les outils utilisés pour le développement, à la fin nous allons présenter quelques interfaces et le résultat obtenu.
\end{itemize}